\documentclass[11pt]{style}

\usepackage[english]{babel}
\usepackage{siunitx}
\usepackage{graphicx}
\usepackage{blindtext}
\usepackage{amsmath,amssymb,units}
\usepackage[outdir=./]{epstopdf}
\memostudent{Ty Davis}
\memocourse{ECE 5420}
\memosubject{Programming Assignment 2 - LZ Compression Algorithm}
\memodate{\today}
\logo{\includegraphics[width=0.5\textwidth]{ece_horiz.pdf}}

\begin{document}
\maketitle

\section{The entropy of the source}

There are 100 0's and 900 1's in the \verb!source1.txt! file. We will use the equation for
entropy listed below:

\[
  H(X) = - \Sigma_{i=1}^{M} p_i \log_2 ( p_i )
\]

We find that 

\[
  H(X) = - 0.1 \log_2 (0.1) - 0.9 \log_2 ( 0.9 ) = 0.469
\]


\section{How many bits are used on average to represent each information bit after compression?}

For the \verb!source1.txt! file, the resulting information was 840 bits long, so
the average compression rate is 0.84.

\section{Is it possible to find a better compression algorithm? Why or why not?}

Yes, the entropy represents the ``theoretical lower
bound of how many bits are required to encode a source
symbol.'' So there is, in theory, a lossless compression
of this data that will take only $\approx470$ bits.

\section{What’s your W number?}

W01367741

\end{document}
